% File: I.tex
% Created: 2014-11-07
% Author: Tesser Paolo
% Email: p.tesser921@gmail.com
%
%
% Modification History
% Version	Modifier Date	Author			Change
% ====================================================================
% 0.0.1		2014-11-07		Tesser-Paolo	inserita sezione
% ====================================================================
% 0.0.2		2015-02-05		Tesser Paolo	inseriti vocaboli: Infrastruttura
% ====================================================================
% 0.0.3		2015-02-11		Tesser Paolo	inseriti vocaboli: Idioma
% ====================================================================
% 0.0.4		2015-03-11		Tesser Paolo	inserito vocabola: Ingegneria dei Requisiti
% ====================================================================
%

\section{I}

\begin{itemize}
	\item \textbf{Idioma}: [PR] è una soluzione specifica ad un linguaggio. Legato quindi alla tecnologia. Esso rinuncia alla genericità della soluzione basata su un design pattern a favore di un implementazione che sfrutta le caratteristiche e le potenzialità del linguaggio;

	\item \textbf{Incremento}: procedere in questo modo significa aggiungere ad un impianto base. Si inserisce, ma non si toglie;


	\item \textbf{Infrastruttura}: [AM] è l'insieme dei servizi offerti sotto la responsabilità dell'amministratore, necessaria allo svolgimento di un progetto. Questa molte volte è trasversale, in parte o totalmente, a più progetti. \'E composta dalle risorse HW (server, rete, postazione di lavoro) e da quelle SW (ambiente di sviluppo, prova, gestione, documentazione);


	\item \textbf{Ingegneria dei requisiti}: [IR] è uno dei primi temi che riguarda i processi di sviluppo (primari). Di solito richiede un ampio insieme di conoscenza. Secondo ISO 12207 non è un processo a se stante, ma un insieme di attività chiave del processo di sviluppo. I requisiti SW sono uno, ma non il solo, dei principali prodotti di tale attività. Le attività necessarie sono quelle di:
		\begin{itemize}
			\item \textbf{Analisi}: consiste nell'analisi dei bisogni e delle fonti, nella classificazione dei requisiti, nella modellazione concettuale del sistema (Use Case), nell'assegnazione dei requisiti a parti distinte del sistema e nella negoziazione con il committente e con i sotto-fornitori;
			\item \textbf{Specifica di verifica e validazione}: consiste nella predisposizione di revisioni interne/esterne e nella predisposizione di prove e dimostrazioni.
		\end{itemize}
	\noindent
	Oltre a queste attività presenti nel processo di sviluppo, vengono coinvolti altri processi:
		\begin{itemize}
			\item \textbf{Documentazione}: processo attraverso il quale si va a redigere lo studio di fattibilità e l'analisi dei requisiti;
			\item \textbf{Gestione e manutenzione dei prodotti}: processo che coinvolge le attività di tracciamento dei requisiti, di impostazione e gestione della configurazione e di gestione dei cambiamenti, cioè l'insieme di attività che consentono di passare da uno stato p a uno stato q, creando consapevolezza e responsabilità;
		\end{itemize}

	\item \textbf{Inspection}: [VV][analisi statica] il suo obiettivo è quello di rilevare la presenza di difetti eseguendo una lettura mirata del codice. Ciò che è consigliato fare è quindi focalizzare la ricerca su dei presupposti (lista di controllo) da seguire quando si è alla ricerca dei possibili errori;

	\item \textbf{Integrazione incrementale}: [VV][analisi dinamica] può essere effettuata in maniera:
		\begin{itemize}
			\item \textbf{Bottom-up}: si sviluppano e si integrano prima le parti con minore dipendenza funzionale e maggiore utilità, risalendo poi l'albero delle dipendenze. Questa strategia riduce il numero di stub necessari al test, ma porta più tardi alla disponibilità di funzionalità di alto livello e quindi a una visione ritardata del prodotto verso il proponente;
			\item \textbf{Top-down}: si sviluppano prima le parti più esterne, quelle poste sulle foglie dell'albero delle dipendenze e poi si scende. Questa strategia comporta l'uso di molti stub, ma integra a partire dalle funzionalità a più alto livello rendendo possibile in breve tempo la visione del prodotto. Viene attuata generalmente quando si usano architetture molto separate.
		\end{itemize}

	\item \textbf{Iterazione}: procedere in questo modo significa operare raffinamenti o rivisitazioni. A differenza della tecnica incrementale, questa può essere distruttiva. \newline
Devo inoltre fissare un numero preciso di iterazioni da compiere e per farlo ha bisogno di condizioni di uscita forti dal ciclo;

\end{itemize}