% File: introduzione.tex
% Created: 2015-02-05
% Author: Tesser Paolo
% Email: p.tesser921@gmail.com
%
%
% Modification History
% Version	Modifier Date	Author			Change
% ====================================================================
% 0.0.1		2015-02-05		Tesser Paolo	inserito cap. introduzione, da stendere
% ====================================================================
% 0.0.2		2015-02-10		Tesser Paolo	stesura di alcune sigle
% ====================================================================
%

\section{Introduzione} % (fold)
\label{sec:introduzione}
Nel vocabolario vengono introdotte a volte, alcune notazioni prima della spiegazione del vocabolo, per specificare l'area di appartenenza di esso all'interno dell'Ingegneria del Software. Ecco il loro significato:
	\begin{itemize}
		\item \textbf{[IS]}: ingegneria del software, usato per i vocaboli generali
		\item \textbf{[AM]}: amministrazione del sistema
		\item \textbf{[PM]}: project management
		\item \textbf{[PR]}: progettazione
		\item \textbf{[QS]}: qualità software
		\item \textbf{[QP]}: qualità di processo
		\item \textbf{[IR]}: ingegneria dei requisiti
		\item \textbf{[VV]}: verifica e validazione
			\begin{itemize}
				\item \textbf{[AS]}: analisi statica
				\item \textbf{[AD]}: analisi dinamica
			\end{itemize}
		\item \textbf{[DOC]}: documentazione
		\item \textbf{[MM]}: metriche e misurazione del software
		\item \textbf{[PROG]}: programmazione
	\end{itemize}

	\subsection{Informativi} % (fold)
	\label{sub:informativi}
		% \begin{itemize}
			% \item \textbf{Ingegneria del Software, Slide Modulo A}: \url{http://www.math.unipd.it/~tullio/IS-1/2014/}
			% \item \textbf{Failure, Error, Fault}: \url{https://vikashazrati.wordpress.com/2008/10/30/fault-failure-error/};
		% \end{itemize}
	% subsection informativi (end)
% section introduzione (end)
