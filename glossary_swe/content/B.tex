% File: B.tex
% Created: 2014-11-07
% Author: Tesser Paolo
% Email: p.tesser921@gmail.com
% 
%
% Modification History
% Version	Modifier Date	Author			Change
% ====================================================================
% 0.0.1		2014-11-07		Tesser Paolo	inserita sezione
% ====================================================================
% 0.0.2		2014-11-10		Tesser Paolo	inserita lista contenente i diversi nomi
% ====================================================================
% 0.0.3		2015-02-05		Tesser Paolo	inseriti vocabili: Baseline
% ====================================================================
%



\section{B}

	\begin{itemize}
		\item \textbf{Baseline}: è un punto di riferimento dal quale calcolare gli scostamenti avvenuti tra una fase e un'altra, serve quindi anche per monitorare l'avanzamento del progetto. Da questo punto non devo retrocedere più. \'E formata da un insieme di configuration item. \newline
		Devono essere fissate a seconda di obblighi contrattuali, esigenze strategiche o esigenze di mercato, ma in qualunque caso non dovrebbero mai essere inferiori al numero di stati del ciclo di vita scelto;
		\item \textbf{Best Practice (migliore prassi)}: modo di fare che per esperienza e per studio si è comportato bene in determinate circostanze, note e specifiche, e abbia mostrato di garantire i risultati migliori;
		\item \textbf{Broken Window Theory}: teoria sociologica, secondo la quale se si impiegano le risorse nella cura e nel rispetto delle cose che già ci sono si ottengono risultati migliori dell'attuazione di norme repressive. Questo perché la gente attua un emulazione del comportamento della comunità;

	\end{itemize}