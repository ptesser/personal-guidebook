% File: R.tex
% Created: 2014-11-07
% Author: Tesser Paolo
% Email: p.tesser921@gmail.com
%
%
% Modification History
% Version	Modifier Date	Author			Change
% ====================================================================
% 0.0.1		2014-11-07		Tesser-Paolo	inserita sezione
% ====================================================================
%
%

\section{R}

\begin{itemize}
	\item \textbf{Responsabile}: è uno dei ruoli di gestione di progetto. Esso rappresenta il progetto di tutto il team presso il fornitore e il committente. Viene quindi accentrata la responsabilità di scelta e approvazione. \newline
Partecipa interamente al ciclo di vita del prodotto e sa sempre (senza bisogno di domandare nulla agli altri membri) a che punto è l'avanzamento delle attività per il raggiungimento degli obiettivi fissati. \newline
Ha quindi responsabilità sulla pianificazione, sulla gestione delle risorse umane tramite coordinamento e relazioni esterne con il committente. \newline
La pianificazione consiste nella definizione delle attività che il team dovrà svolgere e l'assegnamento dei tempi di esecuzione per quelle attività riuscendo a determinarne i costi di attuazione, da fornire poi come preventivo al proponente e come consuntivo nella fase finale. \newline
Ci sono diversi strumenti per lo svolgimento di questo ruolo, come l'utilizzo di Diagrammi di Gantt e di PERT .

	\item \textbf{Ruolo}: è una funzione aziendale assegnata a progetto. E' un servizio che viene svolto per la comunità. Ne esistono di diversi e non tutti ammettono parallelismo. \newline
Ci sono diversi ruoli che si possono raggruppare più macroscopicamente in quattro gruppi:
	\begin{enumerate}
		\item Sviluppo: ha la responsabilità tecnica e realizzativa;
		\item Direzione: ha la responsabilità decisionale (molto importante);
		\item Amministrazione: ha la responsabilità della gestione dei processi;
		\item Qualità: ha la responsabilità di gestione della qualità.
	\end{enumerate}

\end{itemize}
