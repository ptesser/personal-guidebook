% File: A.tex
% Created: 2014-11-07
% Author: Tesser Paolo
% Email: p.tesser921@gmail.com
% 
%
% Modification History
% Version	Modifier Date	Author			Change
% ====================================================================
% 0.0.1	2014-11-07	Tesser-Paolo		inserita sezione 
% ====================================================================
% 
%

\section{A}

\begin{itemize}
	\item \textbf{Amministratore} : è uno dei ruoli di gestione di un progetto. \\
Esso ha il compito di controllare l'ambiente di lavoro e di amministrare le infrastrutture necessarie allo svolgimento del progetto. \\
Deve mettere in pratica ciò che le norme chiedono e attraverso sistemi automatizzati le deve fare eseguire senza renderle troppo ingombranti per gli altri membri. Questo lavoro va fatto in maniera preventiva e proattiva rispetto l'inizio delle attività e del tutto trasparente agli altri componenti. \\
A questa figura fanno quindi capo le configurazioni sui sistemi di versionamento e tutta la documentazione che viene redatta;

	\item \textbf{Analista} : è uno dei ruoli di gestione di un progetto. \\
Esso ha il compito di capire il problema per ottenere i requisiti. Non da la soluzione e spesso non segue fino in fondo la realizzazione del progetto, ma è presente principalmente nella fase iniziale. \\
Per cercare i requisiti segue un approccio top-down. Parte infatti ad analizzare quelli espliciti del proponente fino a scinderli in vari sottogruppi gerarchici, trovandone nel frattempo anche di impliciti (il numero maggiore tra le due tipoologie);


\end{itemize}
