% File: S.tex
% Created: 2014-11-07
% Author: Tesser Paolo
% Email: p.tesser921@gmail.com
% 
%
% Modification History
% Version	Modifier Date	Author			Change
% ====================================================================
% 0.0.1		2014-11-07		Tesser Paolo	inserita sezione
% ====================================================================
% 0.0.2		2015-02-03		Tesser Paolo	agginte voci: Servizio
%

\section{S}

\begin{itemize}
	\item \textbf{Scrum}: TO DO;
	\item \textbf{SEMAT}: Software Engineering Method and Theory è una comunità industriale con lo scopo di migliorare le pratiche dell'ingegneria del software, ridefinendola come una disciplina rigorosa;
	\item \textbf{Servizio}: è un mezzo per fornire valore all'utente, agevolando il raggiungimento dei suoi obiettivi senza doverne sostenere gli specifici costi e rischi;
	\item \textbf{Sistematico}: costante, che fa sempre la stessa cosa in determinate situazioni, applicando un comportamento, che deve tendere ad esser best, ripetitivo;
	\item \textbf{Stakeholder (portatore di interesse)}: è l'insieme di persone coinvolte nel ciclo di vita del SW con influenza sul prodotto. \newline
Possono essere chi usa lo usa, chi lo paga o chi lo sviluppa;

\end{itemize}